\documentclass[../main.tex]{subfiles}

\newcommand{\Zero}{\mathbf{0}}
\newcommand{\AAA}{\mathbf{A}}
\newcommand{\BBB}{\mathbf{B}}
\newcommand{\CCC}{\mathbf{C}}

\begin{document}
  \chapter{Rozwiązanie problemu}
  W tym rozdziale zostaną zaproponowane algorytmy dostosowane 
  do zadanej macierzy \( \AAA \) wraz ze wcześniejszymi wnioskami.

  \section{Zaadaptowany algorytm Gaussa}

  Wobec wcześniejszych wniosków, zaadaptowany algorytm został sformułowany
  jako algorytm~\ref{alg:elim-gauss-adapt}.
  
  \begin{algorithm}
    \caption{Zaadaptowany algorytm eliminacji Gaussa}
    \label{alg:elim-gauss-adapt}
    \begin{algorithmic}[1]
      \Procedure{GaussianElimination}{$A$, $b$}
        \For{$k = 1, 2, \dotsc, n - 1$} \label{alg:elim-gauss-adapt:first-for}
          \For{$i = k + 1, k + 2, \dotsc, k + \ell - k \bmod \ell$}
          \label{alg:elim-gauss-adapt:second-for} 
          \Comment{wniosek~\ref{col:num-steps}} 
            \State $l_{ik} \gets [A]_{ik} / [A]_{kk}$ 
            \For{$j = k + 1, k + 2, \dotsc, \min \{ n, k + \ell \} $}
            \label{alg:elim-gauss-adapt:third-for}  
            \Comment{wniosek~\ref{col:update-elements-kth-step}}
              \State $[A]_{ij} \gets [A]_{ij} - l_{ik} \cdot [A]_{kj}$
            \EndFor
            \State $b_i \gets b_i - l_{ik} \cdot b_k$ 
          \EndFor
        \EndFor
      \EndProcedure
    \end{algorithmic}
  \end{algorithm}

  \subsection{Analiza złożoności obliczeniowej}

  \begin{fact}
    Algorytm~\ref{alg:elim-gauss-adapt} ma złożoność \( O(n \cdot \ell^2 ) \).
  \end{fact}
  \begin{proof}
   Pętla w linii~\ref{alg:elim-gauss-adapt:first-for} wykonuje się dokładnie
    \( n \) razy. Pętle w liniach~\ref{alg:elim-gauss-adapt:second-for}  
    i~\ref{alg:elim-gauss-adapt:third-for} wykonują się co najwyżej \( \ell \)
    razy. Stąd złożoność algorytmu to \( O(n \cdot \ell^2) \).
  \end{proof}

  \section{Zaadaptowany algorytm wyboru}
  Algorytm został opisany jako algorytm~\ref{alg:choose-alg}.
  Zakładamy, że algorytm wyboru będzie uruchamiany w \( k \)-tym kroku
  eliminacyjnym, stąd kopiowanie wiersza zawsze będzie się odbywać
  od \( k \) do co najwyżej \( k + 2 \ell \). Zakładamy, że permutacja
  \( \sigma \) jest zainicjalizowana jako identyczność.
  \begin{algorithm}
    % dodać kopiowanie wiersza
    \caption{Zaadaptowany algorytm wyboru elementu głównego}
    \label{alg:choose-alg}
    \begin{algorithmic}[1]
      \Procedure{Choose}{$A$, $k$, $\sigma$, $\varepsilon$}
%        \State $\sigma \gets \textrm{id}$
%        \For{$k = 1, 2, \dotsc, n$}
          \State $m \gets \argmax \{ [A]_{ik} : k + 1 \le i \le k + \ell - k \bmod \ell \}$ 
          \Comment{wniosek~\ref{col:num-steps}}
          \If{$|a_{mk}| < \varepsilon$}
            \State \Return error
          \Else
            \State $\sigma(m) \leftrightarrow \sigma(k)$
            \For{$i = k, k + 1, \dotsc, \min \{ n, k + 2 \ell \} $} \Comment{zamiana wierszy}
              \State $[A]_{im} \leftrightarrow [A]_{ik}$
              \State $\ell_{im} \leftrightarrow \ell_{ik}$
            \EndFor
            \State $b_m \leftrightarrow b_k$ \Comment{wykonujemy, jeśli został zadany
              wektor prawych stron}
          \EndIf
%        \EndFor
%        \State \Return $\sigma$
      \EndProcedure
    \end{algorithmic}
  \end{algorithm}

  \subsection{Analiza złożoności obliczeniowej}
  
  \begin{fact}
    Algorytm~\ref{alg:choose-alg} wykona się co najwyżej \( O(n \cdot \ell) \).
  \end{fact}
  \begin{proof}
    Podobna argumentacja co przy analizie złożoności dla eliminacji Gaussa.
  \end{proof}

  \subsection{Dostosowanie algorytmu do eliminacji Gaussa}
  By wykonać eliminację Gaussa z częściowym wyborem uruchamiamy
  algorytm wyboru, który zwraca permutację; dla tak spermutowanej
  macierzy uruchamiamy eliminację Gaussa, następnie wektor wynikowy
  zostaje odwrócony za pomocą wygenerowanej odwrotnej permutacji
  przez algorytm wyboru.

  \section{Zaadaptowany rozkład LU}

  Rozkład LU może zostać trywialnie otrzymany przez spamiętywanie mnożników
  oraz wyniku algorytmu eliminacji Gaussa. 

  Rozkład LU z częściowym wyborem jest przeprowadzany tak samo jak
  eliminacja Gaussa z częściowym wyborem, z tym że spamiętujemy również
  permutację. Przy obliczaniu układu za pomocą rozkładu LU, wektor wynikowy
  odwracamy otrzymaną permutacją.

  Wykorzystujemy również zaadaptowane algorytmy obliczania dla układów z macierzą
  trójkątną dolną i górną.

    \begin{algorithm}
      \caption{Zaadaptowany algorytm rozwiązywania układu z macierzą trójkątną górną}
      \label{alg:ut-matrix-adapt}
      \begin{algorithmic}[1]
        \Procedure{UpperTriangleSolve}{$U$, $b$}
          \For{$i = n, n - 1, \dotsc, 1$}
            \State $\Sigma \gets 0$
            \For{$j = i + 1, i + 2, \dotsc, \min \{ n, i + \ell \} $}
            \Comment{wniosek~\ref{col:update-elements-kth-step}}
              \State $\Sigma \gets \Sigma + [U]_{ij} \cdot x_j$
            \EndFor
            \State $x_i \gets \frac{b_i - \Sigma}{[U]_{ii}}$
          \EndFor
          \State \Return $x$
        \EndProcedure
      \end{algorithmic}
    \end{algorithm}

    \begin{algorithm}
      \caption{Zaadaptowany algorytm rozwiązywania układu z macierzą trójkątną dolną}
      \label{alg:lt-matrix-adapt}
      \begin{algorithmic}[1]
        \Procedure{LowerTriangleSolve}{$L$, $b$}
          \For{$i = 1, 2, \dotsc, n$}
            \State $\Sigma \gets 0$
            \For{$j = o(m), o(m) + 1, \dotsc, m - 1$}
            \Comment{wniosek~\ref{col:lower-bound-L-mat}}
              \State $\Sigma \gets \Sigma + [L]_{ij} \cdot x_j$
            \EndFor
            \State $x_i \gets b_i - \Sigma$
          \EndFor
          \State \Return $x$
        \EndProcedure
      \end{algorithmic}
    \end{algorithm}

  Należy zauważyć, że obliczanie \( x_i \) w algorytmie~\ref{alg:lt-matrix-adapt}
  nie używa dzielenia przez element na przękątnej, ponieważ
  diagonala w macierzy \( L \) składa się z samych jedynek.

  \subsection{Analiza złożoności obliczeniowej}
  Rozkład LU ma taką samą złożoność co dostosowany algorytm eliminacji Gaussa.

  \section{Rozwiązywanie układu równań za pomocą zaadaptowanego rozkładu LU}
  By rozwiązać układ równań za pomocą rozkładu LU, najpierw
  otrzymujemy rozkład \( A = LU \) i otrzymujemy,
  że \( LUx = b \). Stąd wyznaczamy \( Ly = b \) 
  i następnie \( Ux = y \). Algorytmy~\ref{alg:ut-matrix-adapt}
  oraz~\ref{alg:lt-matrix-adapt} pozwalają nam na realizację
  tych obliczeń.

  \subsection{Analiza złożoności obliczeniowej}

  \begin{fact} % naprawić referencje do algorytmów
    Algorytm~\ref{alg:ut-matrix-adapt} i~\ref{alg:lt-matrix-adapt} wykona się
    co najwyżej w czasie \( O(n \cdot \ell) \).
  \end{fact}
  \begin{proof}
    Wynika to bezpośrednio stąd, że wewnętrzną pętlę, która wykonuje się
    co najwyżej \( \lambda \ell \) razy dla pewnego \( \lambda > 0 \), wykonujemy
    \( n \) razy. Automatycznie dostajemy tezę.
  \end{proof}

  \chapter{Wyniki i interpretacja}

  Powyższe algorytmy zostały zaimplementowane w języku Julia. Plik zawierający implementację,
  to \verb|blocksys.jl|. Testy wykorzystujące wygenerowane przykłady na stronie internetowej
  kursu zostały napisane w pliku \verb|test.jl|.

  \section{Tabela wyników}

  Program, wraz z rezultatami opisanymi zgodnie z formatem na liście, wygenerował
  również informację na temat czasu działania oraz błędu względnego macierzy
  \( A \) i jej odpowiedniego rozkładu. Wyniki zostały przedstawione w tabeli~\ref{tab:results}.

  \section{Wnioski}

  Rząd błędu rozwiązania waha się od \( -16 \) do \( -14 \), co potwierdza poprawność tego
  algorytmu. Większy błąd występuje przy algorytmie eliminacji Gaussa bez częściowego wyboru,
  ponieważ możemy dzielić elementy, które są dostatecznie bliskie zeru, powodując dalszą
  utratę precyzji.

  Należy zauważyć, że obliczenie rozkładu (de facto bezpośrednie rozwiązywanie układu równań
  za pomocą algorytmu eliminacji) jest kosztowne, jednakże w bezpośrednich obliczeniach
  układu za pomocą wygenerowanego rozkładu jest już wyjątkowo tanie, stąd ta metoda sprawdza się
  gdy chcemy jednorazowo obliczyć rozkład danej macierzy i następnie wykonywać znaczącą liczbę
  obliczeń, by otrzymać rozwiązanie dla tego układu równań.

  \begin{table}
    \label{tab:results}
    \centering
    \begin{tabular}{|l|l|l|l|l|l|}
      \hline 
      \multicolumn{5}{|c|}{$n = 16$} \\\hline 
      & Gauss & Częściowy Gauss & LU & Częściowe LU \\ \hline
      Błąd wzgl. & \texttt{3.764642e-15} & \texttt{5.019079e-16} &
      \texttt{3.764642e-15} &
      \texttt{5.019079e-16} \\ \hline
      Czas (s) & \texttt{0.000130} & \texttt{0.000212} & \texttt{0.000023} & \texttt{0.000019} \\ \hline
      \multicolumn{5}{|c|}{$n = 10000$} \\\hline 
      & Gauss & Częściowy Gauss & LU & Częściowe LU \\ \hline
      Błąd wzgl. & \texttt{6.601103e-14} & \texttt{3.8352702e-16} &
      \texttt{6.601103e-14} &
      \texttt{3.8352702e-16} \\ \hline
      Czas (s) & \texttt{1.568504} & \texttt{2.583332} & \texttt{0.005531} & \texttt{0.008635} \\ \hline
      \multicolumn{5}{|c|}{$n = 50000$} \\\hline 
      & Gauss & Częściowy Gauss & LU & Częściowe LU \\ \hline
      Błąd wzgl. & \texttt{2.985915e-14} & \texttt{4.1456863e-16} &
      \texttt{2.9859158e-14} &
      \texttt{4.145686e-16} \\ \hline
      Czas (s) & \texttt{76.408719} & \texttt{99.723845} & \texttt{0.026061} & \texttt{0.037191} \\ \hline
    \end{tabular}
    \caption{Tabela błędów i czasu obliczania rozwiązania dla układu równań \( Ax = b \)}
  \end{table}
\end{document}
