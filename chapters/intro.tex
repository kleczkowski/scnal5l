\documentclass[../main.tex]{subfiles}

\newcommand{\Zero}{\mathbf{0}}
\newcommand{\AAA}{\mathbf{A}}
\newcommand{\BBB}{\mathbf{B}}
\newcommand{\CCC}{\mathbf{C}}

\begin{document}
  \chapter{Wstęp}
  W poniższym rozdziale będzie opisana natura problemu potrzebna do rozwiązania
  oraz wszelkie obserwacje, które będą prowadzić do rozwiązania problemu.
  
  Zacznijmy najpierw od formułowania problemu.
  \section{Sformułowanie problemu}
    Niech \( \AAA \in M_{n \times n}(\RR) \) będzie macierzą blokową
    o~strukturze~(\ref{eq:matrix-A-def}),
    \begin{equation} \label{eq:matrix-A-def}
      \AAA = \begin{bmatrix}
        \AAA_1 & \CCC_1 & \Zero  & \Zero      & \Zero       & \hdots      & \Zero       \\
        \BBB_2 & \AAA_2 & \CCC_2 & \Zero      & \Zero       & \hdots      & \Zero       \\
        \Zero  & \BBB_3 & \AAA_3 & \CCC_3     & \Zero       & \hdots      & \Zero       \\
        \vdots & \ddots & \ddots & \ddots     & \ddots      & \ddots      & \vdots      \\
        \Zero  & \hdots & \Zero  & \BBB_{v-2} & \AAA_{v-2}  & \CCC_{v-2}  & \Zero       \\
        \Zero  & \hdots & \Zero  & \Zero      & \BBB_{v-1}  & \AAA_{v-1}  & \CCC_{v-1}  \\
        \Zero  & \hdots & \Zero  & \Zero      & \Zero       & \BBB_v      & \AAA_v
      \end{bmatrix}
    \end{equation}
    przy czym \( v = n / \ell \) (zakładamy, że~\( \ell | n \)).
    Macierz \( \AAA_k \in M_{\ell \times \ell}(\RR) \) jest macierzą gęstą,
    macierz \( \BBB_k \in M_{\ell \times \ell}(\RR) \) jest macierzą rzadką
    postaci~(\ref{eq:matrix-Bk-def})
    \begin{equation} \label{eq:matrix-Bk-def}
      \BBB_k = \begin{bmatrix}
        0       & \hdots & 0      & b^k_1     \\
        0       & \hdots & 0      & b^k_2     \\
        \vdots  & \ddots & \vdots & \vdots    \\
        0       & \hdots & 0      & b^k_\ell  \\
      \end{bmatrix}
    \end{equation}
    oraz macierz \( \CCC_k \in M_{\ell \times \ell}(\RR) \) jest macierzą rzadką
    postaci~(\ref{eq:matrix-Ck-def}).
    \begin{equation} \label{eq:matrix-Ck-def}
      \CCC_k = \begin{bmatrix}
        c^k_1   & 0       & 0       & \hdots        & 0         \\
        0       & c^k_2   & 0       & \hdots        & 0         \\
        \vdots  & \ddots  & \ddots  & \ddots        & \hdots    \\
        0       & \hdots  & 0       & c^k_{\ell-1}  & 0         \\
        0       & \hdots  & 0       & 0             & c^k_\ell  \\
      \end{bmatrix}
    \end{equation}
    
    \begin{problem}
      Należy rozwiązać układ równań \( \AAA \mathbf{x} = \mathbf{b} \)
      za~pomocą zaadaptowanego algorytmu eliminacji Gaussa
      oraz rozkładu~LU.
    \end{problem}
    \begin{remark}
      Będziemy oznaczać przez \( [A]_{ij} \) jako wartość elementu macierzy \( A \)
      w \( i \)-tym wierszu i \( j \)-tej kolumnie.
    \end{remark}

    \section{Opis algorytmów}
    
    By mieć odniesienie co do tego, w jaki sposób algorytm eliminacji Gaussa
    upraszcza się przy podanej postaci macierzy, musimy zdefiniować, czym jest
    algorytm eliminacji Gaussa. Algorytm jest opisany dokładnie w~\cite{kielb92}.

    \subsection{Algorytm eliminacji Gaussa}
    Załóżmy, że macierz \( A \in M_{n \times n}(\RR) \) jest macierzą nieosobliwą,
    \( b \in \RR^n \) jest wektorem wyrazów wolnych. Naszym zadaniem
    jest obliczenie układu równań \( Ax = b \).

    W algorytmie eliminacji Gaussa wyróżniamy podstawowy element algorytmu,
    który nazywamy \emph{krokiem eliminacji}.

    W \( k \)-tym kroku eliminacyjnym oblicza się mnożniki eliminacyjne
    \[ l_{ik} = [A^{(k)}]_{ik}/[A^{(k)}]_{kk} \] gdzie \( A^{(k)} \)
    jest macierzą, która jest w \( k \)-tym kroku eliminacyjnym,
    oraz \( k + 1 \le i \le n \).
    Następnie, po obliczeniu mnożnika, przekształcamy macierz,
    otrzymując tym samym macierz \( A^{(k + 1)} \), zgodnie
    ze wzorem~\ref{eq:elimination-step-A}.
    \begin{equation} \label{eq:elimination-step-A}
      [A^{(k + 1)}]_{ij} := \begin{cases}
        0 & \text{dla $k + 1 \le i \le n$ oraz $j = k$} \\
        [A^{(k)}]_{ij} - l_{ik} \cdot [A^{(k)}]_{kj} & \text{dla $k + 1 \le i \le n$ oraz $k + 1 \le
        j \le n$} \\
        [A^{(k)}]_{ij} & \text{dla pozostałych wskaźników}
      \end{cases}
    \end{equation}
    Tym samym przekształca się wektor prawych stron za pomocą wzoru~\ref{eq:elimination-step-b}.
    \begin{equation} \label{eq:elimination-step-b}
      b^{(k + 1)}_i := \begin{cases}
        b^{(k)}_i - l_{ik} \cdot b^{(k)}_k & \text{dla $k + 1 \le i \le n$} \\
        b^{(k)}_i & \text{dla pozostałych wskaźników}
      \end{cases}
    \end{equation}

    Celem \( k \)-tego kroku eliminacyjnego jest wyrugowanie \( n - k \) ostatnich elementów
    \( k \)-tej kolumny. Powtarzając kroki eliminacyjne dla \( 1 \le k \le n-1 \) 
    sprawimy, że otrzymamy macierz rzadką, trójkątną górną.
    Ten proces prowadzi nas do postaci macierzy, którą potem, przy odpowiedniej
    technice, da nam wektor \( x \) w czasie \( O(n^2) \). Algorytm~\ref{alg:elim-gauss}
    opisuje dokładnie eliminację Gaussa.

    \begin{algorithm}
      \caption{Algorytm eliminacji Gaussa}
      \label{alg:elim-gauss}
      \begin{algorithmic}[1]
        \Procedure{GaussianElimination}{$A$, $b$}
          \For{$k = 1, 2, \dotsc, n - 1$} 
            \For{$i = k + 1, k + 2, \dotsc, n$} \Comment{$k$-ty krok eliminacyjny}
              \State $l_{ik} \gets [A]_{ik} / [A]_{kk}$ \Comment{mnożnik eliminacyjny}
              \For{$j = k + 1, k + 2, \dotsc, n$} \Comment{eliminacja i aktualizacja wiersza}
                \State $[A]_{ij} \gets [A]_{ij} - l_{ik} \cdot [A]_{kj}$
              \EndFor
              \State $b_i \gets b_i - l_{ik} \cdot b_k$ \Comment{aktualizacja wektora prawych stron}
            \EndFor
          \EndFor
        \EndProcedure
      \end{algorithmic}
    \end{algorithm}

    Koszt algorytmu eliminacji Gaussa wynosi, dla macierzy gęstych, co najwyżej \( O(n^3) \)
    operacji.

    \subsection{Algorytm rozwiązywania układu z macierzą trójkątną górną}
    
    Po sprowadzeniu macierzy do postaci górno trójkątnej, eliminujemy
    wszystkie elementy, które znajdują się ponad diagonalą. 
    W zwyczajny sposób dzielimy wiersz przez jego element główny
    i odejmujemy z wierszy wyżej odpowiednie wielokrotności
    podzielonego uprzednio wiersza tak, by wyzerować elementy
    nad elementem głównym.

    Odpowiedni algorytm, który został podany na wykładzie,
    jest oznaczony jako algorytm~\ref{alg:ut-matrix}.

    \begin{algorithm}
      \caption{Algorytm rozwiązywania układu z macierzą trójkątną górną}
      \label{alg:ut-matrix}
      \begin{algorithmic}[1]
        \Procedure{UpperTriangleSolve}{$U$, $b$}
          \For{$i = n, n - 1, \dotsc, 1$}
            \State $\Sigma \gets 0$
            \For{$j = i + 1, i + 2, \dotsc, n$}
              \State $\Sigma \gets \Sigma + [U]_{ij} \cdot x_j$
            \EndFor
            \State $x_i \gets \frac{b_i - \Sigma}{[U]_{ii}}$
          \EndFor
          \State \Return $x$
        \EndProcedure
      \end{algorithmic}
    \end{algorithm}

    \begin{example} \label{ex:gauss-elim}
      Załóżmy, że mamy zadaną macierz nieosobliwą~\ref{eq:example-gauss:A-mat}
      oraz wektor prawych stron~\ref{eq:example-gauss:b-vec}.
      \begin{align}
        A &= \begin{bmatrix} \label{eq:example-gauss:A-mat} 
          3 & 4 & 2 \\
          1 & 8 & 7 \\
          2 & 2 & 3
        \end{bmatrix} \\
        b &= \begin{bmatrix} \label{eq:example-gauss:b-vec}
          4 \\
          2 \\
          0
        \end{bmatrix}
      \end{align}
      Zostanie rozwiązany układ równań \( Ax = b \) za pomocą algorytmu eliminacji
      Gaussa. Skonstruujmy macierz rozszerzoną \( [A | b] \) i następnie dokonujmy
      kolejnych kroków eliminacji.

      Dla pierwszego kroku eliminacyjnego \( k = 1 \) obliczamy mnożniki.
      Są to kolejno \( l_{21} = 1 / 3 \) oraz \( l_{31} = 2 / 3 \).
      Do drugiego i trzeciego wiersza dodajemy pierwszy wiersz pomnożony
      przez odpowiednie mnożniki eliminacyjne. Otrzymujemy wtedy
      \begin{equation*}
        \left[ \begin{array}{ccc|c}
          3 & 4             & 2             & 4             \\
          0 & 6\frac{2}{3}  & 6\frac{1}{3}  & \frac{2}{3}   \\
          0 & -\frac{2}{3}  & 1\frac{2}{3}  & -\frac{2}{3}
        \end{array}\right]
      \end{equation*}

      Dla drugiego kroku zaś mamy, że \( l_{32} = \frac{-\frac{2}{3}}{6\frac{2}{3}} 
      = \frac{-\frac{2}{3}}{\frac{20}{3}} = -\frac{1}{10} \). Przekształcając
      trzeci wiersz odpowiednio, otrzymujemy
      \begin{equation*}
        \left[ \begin{array}{ccc|c}
          3 & 4             & 2             & 4             \\
          0 & 6\frac{2}{3}  & 6\frac{1}{3}  & \frac{2}{3}   \\
          0 & 0             & 2\frac{3}{10} & -\frac{3}{5}
        \end{array}\right]
      \end{equation*}
      
      Uzyskaliśmy właśnie macierz trójkątną górną. Teraz rozwiązujemy
      układ z macierzą trójkątną górną, zaczynając od ostatniego wiersza do
      pierwszego. Ostatni wiersz dzielimy przez element na przekątnej,
      następnie odejmujemy wielokrotność ostatniego wiersza od pozostałych
      wierszy tak, aby wyeliminować elementy nad przekątną.
      \begin{equation*}
        \left[ \begin{array}{ccc|c}
          3 & 4             & 0             & \frac{104}{23}   \\
          0 & 6\frac{2}{3}  & 0             & \frac{160}{69}   \\
          0 & 0             & 1             & -\frac{6}{23}
        \end{array}\right]
      \end{equation*}
      Podobnie robimy z drugim wierszem. 
      \begin{equation*}
        \left[ \begin{array}{ccc|c}
          3 & 0             & 0             & -\frac{536}{23}  \\
          0 & 1             & 0             & \frac{160}{23}   \\
          0 & 0             & 1             & -\frac{6}{23}
        \end{array}\right]
      \end{equation*}
      Po podzieleniu pierwszego wiersza przez trzy, otrzymujemy rozwiązanie:
      \begin{equation*}
        \left[ \begin{array}{ccc|c}
          1 & 0             & 0             & -\frac{536}{69}  \\
          0 & 1             & 0             & \frac{160}{23}   \\
          0 & 0             & 1             & -\frac{6}{23}
        \end{array}\right]
      \end{equation*}
    \end{example}

    \subsection{Algorytm rozwiązywania układu z macierzą trójkątną dolną}

    By móc przeprowadzić rozwiązywanie równań po rozkładzie LU, potrzebny
    jest nam algorytm wyznaczający rozwiązanie układu równań z macierzą
    trójkątną dolną. Algorytm jest analogiczny co~\ref{alg:ut-matrix},
    jest podany jako~\ref{alg:lt-matrix}.

    \begin{algorithm}
      \caption{Algorytm rozwiązywania układu z macierzą trójkątną dolną}
      \label{alg:lt-matrix}
      \begin{algorithmic}[1]
        \Procedure{LowerTriangleSolve}{$L$, $b$}
          \For{$i = 1, 2, \dotsc, n$}
            \State $\Sigma \gets 0$
            \For{$j = 1, \dotsc, i - 1$}
              \State $\Sigma \gets \Sigma + [L]_{ij} \cdot x_j$
            \EndFor
            \State $x_i \gets \frac{b_i - \Sigma}{[L]_{ii}}$
          \EndFor
          \State \Return $x$
        \EndProcedure
      \end{algorithmic}
    \end{algorithm}

    \subsection{Rozkład LU}
    Rozkład LU to zapisanie macierzy \( A \) za pomocą macierzy trójkątnej dolnej
    i górne, przy czym zachodzi \( A = LU \).

    Zgodnie z~\cite{kielb92} wykonanie rozkładu LU sprowadza się 
    do zaimplementowania algorytmu eliminacji Gaussa
    na macierzy, bez rozwiązywania układu równań. Efektem ubocznym algorytmu są mnożniki
    eliminacyjne, które formują nam macierz \( L \) jako macierz trójkątną dolną
    i rezultat algorytmu jako macierz trójkątną górną \( U \). Wtedy \( A = LU \),
    dla pewnej macierzy \( A \in M_{n \times n}(\RR) \). Dokładnie ujmując, macierz
    \( L_A \) jest postaci opisanej wzorem~\ref{eq:lower-matrix}, gdzie \( l_{ik} \)
    to mnożniki eliminacyjne.
    \begin{equation} \label{eq:lower-matrix}
      [L_A]_{ij} = \begin{cases}
        l_{ij} & \text{dla $i < j$} \\
        1 & \text{dla $i = j$} \\
        0 & \text{dla pozostałych wskaźników}
      \end{cases}
    \end{equation}
    Natomiast macierz \( U_A \) jest macierzą, będącą rezultatem działania algorytmu
    eliminacji Gaussa.

    \begin{example}
      Macierz \( L \) z przykładu~\ref{ex:gauss-elim} możemy zapisać jako~\ref{eq:lu-ex:L-mat}
      oraz macierz \( U \) jako~\ref{eq:lu-ex:U-mat}.
      \begin{align} \label{eq:lu-ex:L-mat}
        L &= \begin{bmatrix}
          1 & 0 & 0 \\
          \frac{1}{3} & 1 & 0 \\
          \frac{2}{3} & -\frac{1}{10} & 1 
        \end{bmatrix} \\
        U &= \begin{bmatrix} \label{eq:lu-ex:U-mat}
          3 & 4 & 2 \\
          0 & 6\frac{2}{3} & 6\frac{1}{3} \\
          0 & 0 & 2\frac{3}{10}
        \end{bmatrix}
      \end{align}
    \end{example}

    \subsection{Wybór elementu głównego}
    Nie zawsze element leżący na diagonali jest elementem, który można wybrać do obliczenia
    mnożnika eliminacyjnego, ponieważ może nastąpić dzielenie przez zero. Stąd, by algorytm
    zachowywał się poprawnie dla macierzy, które są nieosobliwe, a mają takie elementy, 
    modyfikuje się algorytm eliminacji Gaussa przez wybór elementu głównego. Na potrzeby
    rozwiązania będzie nas interesować wariant wyboru znany jako wybór częściowy.

    Wybór częściowy odbywa się na tym, że szukamy największego elementu wśród 
    \( k \)-tej kolumny podczas \( k \)-tego kroku eliminacyjnego. Oznaczmy ten element jako
    \( M_k = \max \{ [A]_{kk}, [A]_{(k+1)k}, \dotsc, [A]_{nk} \} \). Jeśli \( M_k \) jest
    dostatecznie małe, oznacza to, że liczona macierz może być prawdopodobnie macierzą
    osobliwą. Załóżmy, że \( M_k = [A]_{\beta k} \) dla pewnego \( \beta \in \NN \).
    Jeżeli \( M_k \) nie jest dostatecznie małe, to zamieniamy wiersz \( \beta \) z wierszem \(k \).
    Powyższą inwersję zapamiętujemy w tablicy, która
    będzie interpretowana jako permutacja na \( \{ 1, 2, \dotsc, n \} \).

    Odwrócenie permutacji jest proste, wymaga \( O(n) \) operacji, stąd przywrócenie
    poprawnej kolejności wiąże się jedynie z nałożeniem odwrotnej permutacji na wektor.

    Powyższą modyfikację również nanosi się na rozkład LU.

    \begin{example}
      Załóżmy, że mamy do czynienia z macierzą~\ref{eq:pivoting-ex:A-mat}. Zauważmy, że
      macierz \( A \) na diagonali nie ma elementów niezerowych w pełni, stąd 
      będziemy zmuszeni użyć algorytmu wyboru częściowego.
      \begin{equation} \label{eq:pivoting-ex:A-mat}
        A = \begin{bmatrix}
          1   & 0   & 2 \\
          0   & 0   & 4 \\ 
          8   & 8   & 0 
        \end{bmatrix}
      \end{equation}

      Elementy pod pierwszym elementem na diagonali mogą zostać wyeliminowane
      tak jak do tej pory. Wtedy otrzymujemy macierz~\ref{eq:pivoting-ex:after-first}.
      \begin{equation} \label{eq:pivoting-ex:after-first}
        \begin{bmatrix}
          1 & 0 & 2   \\
          0 & 0 & 4   \\
          0 & 8 & -16 \\
        \end{bmatrix}
      \end{equation}
      W drugim kroku element na diagonali jest zerowy (numerycznie przyjmujemy,
      że element jest dostatecznie bliski zeru). Wobec tego wybieramy największy
      element w tej kolumnie. Jest to \( 8 \). Zamieniamy wiersze tak, by
      na diagonali pojawiła się liczba \( 8 \). Reszta algorytmu odbywa się tak
      jak w zwykłej eliminacji Gaussa.
    \end{example}

\end{document}
