\documentclass[../main.tex]{subfiles}

\newcommand{\Zero}{\mathbf{0}}
\newcommand{\AAA}{\mathbf{A}}
\newcommand{\BBB}{\mathbf{B}}
\newcommand{\CCC}{\mathbf{C}}

\begin{document}
  \chapter{Wstęp}

  \section{Sformułowanie problemu}
    Niech \( \AAA \in M_{n \times n}(\RR) \) będzie macierzą blokową
    o~strukturze~(\ref{eq:matrix-A-def}),
    \begin{equation} \label{eq:matrix-A-def}
      \AAA = \begin{bmatrix}
        \AAA_1 & \CCC_1 & \Zero  & \Zero      & \Zero       & \hdots      & \Zero       \\
        \BBB_2 & \AAA_2 & \CCC_2 & \Zero      & \Zero       & \hdots      & \Zero       \\
        \Zero  & \BBB_3 & \AAA_3 & \CCC_3     & \Zero       & \hdots      & \Zero       \\
        \vdots & \ddots & \ddots & \ddots     & \ddots      & \ddots      & \vdots      \\
        \Zero  & \hdots & \Zero  & \BBB_{v-2} & \AAA_{v-2}  & \CCC_{v-2}  & \Zero       \\
        \Zero  & \hdots & \Zero  & \Zero      & \BBB_{v-1}  & \AAA_{v-1}  & \CCC_{v-1}  \\
        \Zero  & \hdots & \Zero  & \Zero      & \Zero       & \BBB_v      & \AAA_v
      \end{bmatrix}
    \end{equation}
    przy czym \( v = n / \ell \) (zakładamy, że~\( \ell | n \)).
    Macierz \( \AAA_k \in M_{\ell \times \ell}(\RR) \) jest macierzą gęstą,
    macierz \( \BBB_k \in M_{\ell \times \ell}(\RR) \) jest macierzą rzadką
    postaci~(\ref{eq:matrix-Bk-def})
    \begin{equation} \label{eq:matrix-Bk-def}
      \BBB_k = \begin{bmatrix}
        0       & \hdots & 0      & b^k_1     \\
        0       & \hdots & 0      & b^k_2     \\
        \vdots  & \ddots & \vdots & \vdots    \\
        0       & \hdots & 0      & b^k_\ell  \\
      \end{bmatrix}
    \end{equation}
    oraz macierz \( \CCC_k \in M_{\ell \times \ell}(\RR) \) jest macierzą rzadką
    postaci~(\ref{eq:matrix-Ck-def}).
    \begin{equation} \label{eq:matrix-Ck-def}
      \CCC_k = \begin{bmatrix}
        c^k_1   & 0       & 0       & \hdots        & 0         \\
        0       & c^k_2   & 0       & \hdots        & 0         \\
        \vdots  & \ddots  & \ddots  & \ddots        & \hdots    \\
        0       & \hdots  & 0       & c^k_{\ell-1}  & 0         \\
        0       & \hdots  & 0       & 0             & c^k_\ell  \\
      \end{bmatrix}
    \end{equation}
    
    \begin{problem}
      Należy rozwiązać układ równań \( \AAA \mathbf{x} = \mathbf{b} \)
      za~pomocą zaadaptowanego algorytmu eliminacji Gaussa
      oraz rozkładu~LU.
    \end{problem}
\end{document}
