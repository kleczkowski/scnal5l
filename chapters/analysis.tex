\documentclass[../analysis.tex]{subfiles}

\newcommand{\Zero}{\mathbf{0}}
\newcommand{\AAA}{\mathbf{A}}
\newcommand{\BBB}{\mathbf{B}}
\newcommand{\CCC}{\mathbf{C}}
\begin{document}
  \section{Analiza i wnioski}

  W tej sekcji przyjrzyjmy się macierzy~\ref{eq:matrix-A-def}, 
  która została sformułowana wcześniej.

  \subsection{Obserwacje dotyczące algorytmu eliminacji Gaussa}

  Macierz ta jest niewątpliwie rzadka, stąd naiwne stosowanie
  algorytmu Gaussa i rozkładu LU przyniosłoby duży narzut
  czasowy, stąd potrzebne są pewne uproszczenia, które spowodują,
  że czas rozwiązania będzie co najwyżej \( O(n) \).

  \begin{observation}
    Kiedy należy wyeliminować elementy z \( k \)-tej kolumny,
    to trzeba wyeliminować dokładnie \( \ell - k \bmod \ell \)
    elementów pod \( k \)-tym elementem diagonali.
  \end{observation}
  \begin{example}
    Niech \( \ell = 3 \) oraz \( n = 6 \). Mamy macierz 
    postaci~\ref{eq:example-AAA}. Przekreślone elementy
    to te, które chcemy w poszczególnych krokach wyeliminować.
    \begin{equation} \label{eq:example-AAA}
      \AAA = \begin{bmatrix}
        a_{11}            & a_{12}          & a_{13}          & c_{1}           &                 &        \\
        \cancel{a_{21}}   & a_{22}          & a_{23}          &                 & c_{2}           &        \\
        \cancel{a_{31}}   & \cancel{a_{32}} & a_{33}          &                 &                 & c_{3}  \\
                          &                 & \cancel{b_{1}}  & d_{11}          & d_{12}          & d_{13} \\
                          &                 & \cancel{b_{2}}  & \cancel{d_{21}} & d_{22}          & d_{23} \\
                          &                 & \cancel{b_{3}}  & \cancel{d_{31}} & \cancel{d_{23}} & d_{33} 
      \end{bmatrix}
    \end{equation}

    Należy zauważyć, że liczba elementów do wyeliminowania tworzy ciąg \( 2, 1, 3, 2, 1, 3, 2, 1,
    \dotsc \). Przez indukcję dostajemy, że ten ciąg jest zadany wzorem \( \ell - k \bmod \ell \).
  \end{example}
  Dostajemy oczywiście prosty wniosek wynikający z natury algorytmu Gaussa.
  \begin{corollary} \label{col:num-steps}
    Należy wyeliminować w \( k \)-tym kroku eliminacyjnym
    elementy kolumny od \( k + 1 \), do \( k + \ell - k \bmod \ell \).
  \end{corollary}
  Również zauważamy, że
  \begin{observation} 
    By zmodyfikować wiersz, w którym wyeliminowaliśmy pewien element,
    potrzebujemy zmodyfikować co najwyżej \( \ell \) elementów 
    w wierszu.
  \end{observation}
  \begin{example}
    Weźmy macierz~\ref{eq:example-AAA} jako przykład. Za skalarami
    \( c_k \) są tylko same zera, wobec czego, od 
    \( [\AAA]_{kk} \) do \( c_k \) jest \( \ell + 1 \) elementów 
    w wierszu. Eliminując elementy leżące pod przekątną dla \( m > k \)
    mamy, że \( [\AAA]_{mk} = 0 \) a dla każdego takiego kroku
    od \( [\AAA]_{mp} \), gdzie \( p > k \), do pewnego \( c_q \),
    gdzie \( q \in \mathbb{N} \), jest dokładnie \( \ell \) elementów
    do zaktualizowania. Nie możemy pominąć zer pod skalarami z macierzy
    \( \CCC_i \), ponieważ w trakcie kroku eliminacyjnego, zera mogą stać się 
    niezerowe pod diagonalą. 

    Oczywiście dbamy o zakres tablicy, stąd bierzemy zawsze minimum
    z wymiaru i przesunięcia o \( \ell \) miejsc względem \( k \).
  \end{example}
  Podobnie jak w poprzedniej obserwacji otrzymujemy ważny, oraz dość prosty, wniosek:
  \begin{corollary} \label{col:update-elements-kth-step}
    Należy aktualizować elementy od \( k + 1 \) do \( \min \{ n, k + \ell \} \)
    w \( k \)-tym kroku eliminacyjnym.
  \end{corollary}

  \subsection{Uproszczenie algorytmu wyboru}

  Algorytm wyboru musi spełniać ważną własność --- algorytm nie może naruszać
  struktury macierzy. Co więcej musimy ustalić, które indeksy w kolumnach
  możemy wybierać, by otrzymać element główny. Wniosek~\ref{col:num-steps}
  zapewnia nam, jakie elementy możemy wybierać, ponieważ wybieramy
  zawsze elementy pod diagonalą. Jednakże będziemy musieli aktualizować
  \( 2 \ell \) elementów w wierszu w trakcie kroku eliminacyjnego, 
  ponieważ zaburzyliśmy strukturę macierzy \( \CCC_i \) i w najgorszym 
  wypadku, możemy przestawić strukturę \( \CCC_i \) tak, że zamieniamy
  pierwszy wiersz z ostatnim wierszem.
  
  Należy również się zastanowić, ile elementów wiersza będziemy musieli kopiować.
  Zgodnie z poprzednią obserwacją należy kopiować \( 2 \ell + 1 \) elementów.

  \subsection{Uproszczenie algorytmu rozwiązywania układu z macierzą trójkątną górną}
  Wniosek~\ref{col:update-elements-kth-step} przenosi się bezpośrednio
  na algorytm rozwiązywania układu równań z macierzą trójkątną górną,
  ponieważ tylko tam algorytm w krokach eliminacyjnych mógł
  pozostawić jakieś niezerowe wartości.

  \subsection{Uproszczenie algorytmu rozwiązywania układu z macierzą trójkątną dolną}
  \begin{observation}
    Dla dowolnego \( k, m \in \mathbb{N} \), \( k > 0 \)
    i \( k\ell < m \le (k + 1) \ell \) \( m \)-ty wiersz mnożników eliminacyjnych 
    w macierzy \( L \)
    zaczyna się od \(k \ell \)-tej kolumny i kończy się na
    \( m \)-tej kolumnie. Jeśli \( m \le \ell \), to wiersz zaczyna się w pierwszej kolumnie
    i kończy w \( m \)-tej kolumnie.
  \end{observation}
  \begin{example}
    Po uruchomieniu algorytmu eliminacji Gaussa dla macierzy o parametrach \( \ell = 3 \)
    oraz \( n = 6 \) otrzymujemy następującą macierz~\ref{eq:example-L-mat}.

    \begin{equation} \label{eq:example-L-mat}
      L = \begin{bmatrix}
        l_{11}    &         &         &         &       & \\
        l_{21}    & l_{22}  &         &         &       & \\
        l_{31}    & l_{32}  & l_{33}  &         &       & \\
                  &         & l_{43}  & l_{44}  &       & \\
                  &         & l_{53}  & l_{54}  & l_{55}& \\
                  &         & l_{63}  & l_{64}  & l_{65}& l_{66}
      \end{bmatrix}
    \end{equation}

    Zauważamy, że indeks kolumny pierwszego elementu w \( m \)-tym wierszu
    tworzy ciąg \( (1, 1, 1, 3, 3, 3, 6, 6, 6) \) zależny od \( m \), co potwierdza
    naszą obserwację.
  \end{example}
  \begin{corollary} \label{col:lower-bound-L-mat} 
    Indeksem kolumny pierwszego elementu \( m \)-tego wiersza macierzy \( L \) nazywamy liczbę
    \( o(m) \) zadaną wzorem~\ref{eq:lower-bound-L-mat}.
    \begin{equation} \label{eq:lower-bound-L-mat}
      o(m) = \max \left\{ 1, \left( \floor*{\frac{m - (\ell + 1)}{\ell}} + 1 \right) \ell \right\}
    \end{equation}
  \end{corollary}
\end{document}
